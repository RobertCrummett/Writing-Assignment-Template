%% Graduate school personal statement
\documentclass[12pt]{article}
\usepackage{fontspec}

%----------------------------------
%% Document Settings
\pagestyle{empty} % No page numbers
\setmainfont{Times New Roman} % Font

%----------------------------------
%% Functions
% Title
\renewcommand{\title}
{
  \noindent {\LARGE Action Plan} \\
  {\normalsize Practical Steps to Achieve Wellness Goals} \\
  {\small R. Nate Crummett, \today} \\
  \noindent \makebox[\linewidth]{\rule{\textwidth}{0.8pt}} \\ % horizontal line across page
}

% Section title
\newcommand{\sectiontitle}[1]{ \noindent \textbf{\large #1} \\[-4mm] }

%----------------------------------
%% Document
\begin{document}
  \title \\ [-1mm]  
  \sectiontitle{Improving My Communication Skills}

  Improving my communication skills is my first major wellness goal. In my wellness
  vision, I stated that I would especially like to improve my conversations with my advisor.
  This involves both aspects of responding well and listening well. \\

  In order to improving my listening, I will work on being focused during conversations.
  Recently a LinkedIn post alerted me that it has become socially normal to
  \textit{multitask} during meetings. Practically, I can remove distractions during meetings
  and increase the amount of information I absorb by:

  \begin{enumerate}
    \item Closing my laptop when a meeting begins
    \item Putting my phone on do not disturb
    \item Writing down notes during a meeting, to help me keep focused
  \end{enumerate}

  I am already good about not being distracted by screens in the classroom setting, but I have work
  to do elsewhere. The third point about writing down notes is a tactic I recently started employing,
  and have found it helps me track with a speaker's argument. Another advantage I have found is that
  the notes become reference material afterwards that remainds me of the critical ideas, questions,
  and takeaways. I would like to start taking notes more regularly in the personal
  meeting setting. Enacting this will help me be a better listener, particulary with my
  advisor. \\

  The way I talk is the other aspect of improving my communication skills. I understand that I can
  always get better in this category by thinking more before I speak, choosing language appropriate
  for the setting, etc. But even though it is not difficult to think about ways to improve
  my discourse, acting on these broad (vague) ideas is not so simple. Fortunately for me, I am
  working for an advisor who has thought about communication alot, and has no trouble telling me
  where I can improve. What I can do with his (and other people's) critism is \\
  \pagebreak
  \begin{enumerate}
    \item listen and understand the criticsm
    \item pin-point exactly what I said that made people speak up
    \item ask for help improving my speech / presentation / sales pitch / question etc
  \end{enumerate}

  The first point may seem too obvious to list, but I think it is important. It is easy for me
  to take critism as a personal attack and ignore it or fight against it. My people taught me better,
  so when I receive critism, I need to do my best to listen. Occasionally I still feel the
  fight well up in when I am challenged on how I express myself. Being a better listener will 
  make me a better speaker overall. \\

  Finally, asking for help will help me become a better communicator. Sometimes I recieve
  criticism from someone so much more experienced than myself that I simply \underline{
  do not understand} why I am being criticized. Asking the 'why' and 'what exactly' type of
  questions will not only shows me better what I can improve, but will help me become a better
  communicator in general. I have to be tactiful about how I do this though, because I have
  realized such questions can easily be mistaken for fighting words. This is the final practical
  step, because in order to do it well I have to both listen well and not get emotionally
  uptight before I ask for clarification. \\

  \sectiontitle{Places To Go}

  I would like to go see some of the midwest while I am stationed here in Denver. An
  easy way to accomplish this wellness goal is take a weekend off and go for a drive.
  However, a busy school schedule combined with research deadlines makes time tight.
  Too tight for leasurely weekends in my experience thus far. What are practical measures
  I can take that will allow me to travel more?

  \begin{enumerate}
    \item time management
    \item planning trips in advance
  \end{enumerate}
  
  Time management involves getting my work done of course. But something I learned in this course
  is to start saying no to things that are not important to me. Alot of my time has been spent
  in pursuit of things that are not very important to me because I tend to be a yes-man in the
  work environment. But every yes is really a no, and I do not want to say no to an occasional
  weekend off anymore. \\

  I am more prone to commit to time management when I have well established plans for travel.
  I am pleased to say I am in a lab where it is not frowned upon to take a weekend off. In fact,
  my advisor encourages getting out of here on the weekends, and appreciates it when we let
  him know we will be out of touch for certain periods of time. I have taken advantage of this 
  for trips home during the holidays before, but never for a weekend roadtrip. This is a practical
  strategy to get some time off in the great outdoors. \\

\end{document}
